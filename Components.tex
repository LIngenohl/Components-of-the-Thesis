\documentclass[12pt]{article}
	\usepackage[T1]{fontenc}
	\usepackage[utf8]{inputenc}
	\usepackage[british]{babel}
	\usepackage[a4paper]{geometry}
	\geometry{verbose,tmargin=3cm,bmargin=3.5cm,lmargin=4cm,rmargin=3cm,marginparwidth=70pt}
	\setcounter{secnumdepth}{3}
	\setcounter{tocdepth}{3}
	\usepackage{prettyref}
	\usepackage{textcomp}
	\usepackage{setspace}
	\usepackage{indentfirst}
	\usepackage{fancyhdr}
	\usepackage{url}
	\usepackage[normalem]{ulem}
	\usepackage[table, fixpdftex]{xcolor}
	\usepackage{algpseudocode}
	\usepackage{bigstrut}
	\usepackage{enumitem}

	% package hyperref
    \usepackage[hidelinks]{hyperref}
  

	% biblatex
	\usepackage[style=authoryear,natbib=true,maxcitenames=2, maxbibnames=11,backend=biber,pagetracker=page,hyperref=true]{biblatex} \usepackage{csquotes}
	\renewcommand*{\bibsetup}{%
		\interlinepenalty=10000\relax % default is 5000
		\widowpenalty=10000\relax
		\clubpenalty=10000\relax
		\raggedbottom
		\frenchspacing
        \biburlsetup}
        
	% fixes the page number of the first page of each chapter
	\fancypagestyle{plain}{
			\fancyhead{}
			\renewcommand{\headrulewidth}{0pt}
			\renewcommand{\footrulewidth}{0pt}
			\fancyfoot[OC]{\begin{flushright}\thepage\end{flushright}}
    }
    
	% fancy headers for the thesis
	\fancyhead{}
	\fancyhead[RO]{\slshape \nouppercase \rightmark}
	\fancyfoot[OC]{\begin{flushright}\thepage\end{flushright}}
	\renewcommand{\headrulewidth}{0.4pt}
	\setlength{\headheight}{14pt}

	% add bibliography database
	\addbibresource{BA Kopie.bib}
	
	% space between biblio items
	\setlength\bibitemsep{1.7\itemsep} 
	
	% title without ""
	\DeclareFieldFormat[inbook]{title}{#1}
	% non-italic
	\DeclareFieldFormat[online]{tlaitle}{#1} 
	% title unquoted
	\DeclareFieldFormat[article]{title}{#1} 
	% no pp. 
	\DeclareFieldFormat[article]{pages}{#1} 
	% bold volume
	\DeclareFieldFormat*{volume}{\mkbibbold{#1}\setpunctfont{\textbf}}
	
	% no in:
	\renewbibmacro{in:}{} 
	
	% (volume)
	\renewbibmacro*{volume+number+eid}{%
			\printfield{volume}%
			%\setunit*{\adddot}% DELETED
			% \setunit*{\addnbspace}% NEW (optional); there's also \addnbthinspace
			\printfield{number}%
			% \setunit{\addcomma\space}%
			\printfield{eid}}
	\DeclareFieldFormat[article]{number}{\mkbibparens{#1}} 
	
	% edition.
	\DeclareFieldFormat{edition}%
	{(\ifinteger{#1}%
			{\mkbibordedition{#1}\addthinspace{}ed.}%
			{#1\isdot}).}
	
	% publisher and location position
	\renewbibmacro*{publisher+location+date}{%
			\printlist{publisher}%
			\setunit*{\addcomma\space}%
			\printlist{location}%
			\setunit*{\addcomma\space}%
			\usebibmacro{date}%
			\newunit}
	
	% shortauthor before author
	\renewbibmacro*{begentry}{%
			\ifkeyword{Key}{\sffamily}{}%
			\iffieldundef{shorthand}
			{}
			{\global\undef\bbx@lasthash
					\printfield{shorthand}%
					\addcolon\space}%
			\ifboolexpr{test {\usebibmacro{bbx:dashcheck}} or test {\ifnameundef{shortauthor}}}%
			{}%
			{\printnames{shortauthor}%
                    \addspace\textendash\space}}
                    
\title{Components of the Thesis}
\author{Leopold Ingenohl}


\begin{document}
\maketitle

\pagebreak

\tableofcontents

\pagebreak

\section{Motivation \& Importance} 

\subsection{Investor Activism}

    \begin{itemize}
        
        \item Importance - who are the targets of activism? \citep{CoffeeJr.2014}
        
        \item Within the sample of 13D filings, some of the acquirers are corporations that are potential fullacquirers, while other acquirers are institutional investors that are not likely to pursue a complete takeover. \citep{Brigida2012}

        \item We find that the market reacts favorably to activism, consistent with the view that it creates value. The filing of a Schedule 13D revealing an activist fund’s investment in a target firm results in large positive average abnormal returns, in the range of 7\% to 8\%, during the (–20,+20) announcement window \citep{Brav2008}

        \item Filing a Schedule 13D allows the investor to behave in an active manner. \citep{Brigida2012}

        \item these activist shareholders know they can increase the value of the firm they invest in by their own effort (e.g., shareholder activism).\citep{Collin-Dufresne2015}

        \item In the spirit of Pound (1992), we define an entrepreneurial activist as an investor who buys a large stake in a publicly held corporation with the intention to bring about change and thereby realize a profit on the investment \citep{Klein2009}

        \item is still largely unanswered where the announcement premium (and the upward drift in stock prices thereafter, for that matter) comes from \citep{Greenwood2009}

        \item starting with Mikkelson and Ruback (1985), note that mergers and takeovers are often preceded by the acquisition of a minority stake in the target. \citep{Greenwood2009}

        \item define an activist investor who tries to change the status quo through ‘voice’, without a change in control of the firm \citep{Greenwood2009}

        \item To the extent that strategic corporate bidders are perceived to be less likely to abandon their pursuit of the full acquisition, the acquisition probability for corporate partial bidders should be higher than non-corporate bidders.

    \end{itemize}

\subsection{Anecdotal Reference}

    \begin{itemize}
        \item HNA
        \item out of the 2005 merger between retail giants Sears an battle emerged over two little letters: D and G.1 ESL Partn investment entity controlled by Kmart Chairman Edward L hit with a class action lawsuit in which plaintiffs alleged ESL's of its newly acquired ownership position in Sears on a Schedul stead of a Schedule 13 D) was \citep{Giglia2018}

        \item The acquisition of a partial stake in a target firm has been positively linked to the likelihood that the target will be involved in a follow on full acquisition involving either the original bidder or a third party bidder \citep{Akhigbe2007}

    \end{itemize}

\subsection{13D Filings}

    \begin{itemize}
        \item Much attention has recently been given to the current Securities and Exchange Commission reporting requirements for Schedule 13D, the beneficial ownership form many investors must file to report their equity hold \citep{Giglia2018}

        \item Schedule 13D filings must be made within 10 days of acquiring a beneficial ownership of 5\% or greater of the outstanding common stock of a U.S. public company. The use of the qualifier ‘beneficial’ is important because related, yet different entities, may have to file a schedule 13D if their combined ownership of the target is 5\% or greater and their voting or investment power is combined \citep{Brigida2012}

        \item Within the Schedule 13D and 13G filings is information important to this analysis. \citep{Brigida2012}

        \item \emph{Outlook}: with more information processed one could say... A 13D filing by an acquirer may have a more pronounced impact if the filing specifies that the investor intends to be an activist. \citep{Brigida2012}

        \item 13D Filings (Collin-Dufresne, Pierre; Fos, Vyacheslav)

        \item Those investors with activist intentions must file a more detailed Schedule 13D, which along with other information, requires the investor to state its future intentions with respect to influencing control of the company \citep{Giglia2018}

        \item Exchange Act of 1934 (1934 Act)16 in an attempt to increase regulation of tender offers and accumulations of stock. There were no corresponding regulations in connection with cash tender offer \citep{Giglia2018}.

        \item Of relevance here is section 13(d), which governs disclosures of beneficial ownership interests in excess of five percent of certain classes of equity securities. 

        \item Purpose of the filing -- Instead, the purpose of the section focused on informing investors about purchases of large blocks of shares acquired in a short period of time by individuals who could then influence or change control of the issuing company \citep{Giglia2018}

        \item Most obviously, disclosing a large buying interest in the market may push stock prices up, as market makers reacts to this increasing demand by raising ask prices. \citep{Giglia2018}
        
        \item A Schedule 13D/A is an amended filing by the same investor for the same firm and is filed subsequent to the original Schedule 13D \citep{Klein2009}
    \end{itemize}

\subsection{Abnormal Returns}
    \begin{itemize}
        \item An average Schedule 13D filing in our sample is characterized by a positive and significant market reaction upon announcement \citep{Collin-Dufresne2015}

        \item Short-horizon event studies of stock returns: Many studies have examined what happens to targets firm’s stock price when there is a Schedule 13D filing with the SEC \citep{CoffeeJr.2014}

        \item Brav, Jiang, Partnoy, and Thomas (2008) have documented a positive and significant average abnormal return in response to 13D filings \citep{Brigida2012}

        \item 10K-Finlings - While abnormal trading volume and return volatility may indicate market reaction to new information, they could also simply reflect an increase in noise trading. \citep{You2009}

        \item the high returns documented around the announcement of activism reflect investors’ expectations that target firms will be acquired at a premium to the current stock price \citep{Greenwood2009}

        \item We find that PTs experience significant gains when the partial acquisition is announced. \citep{Akhigbe2007}
        
        \item The gains are greater when corporate bidders ini- tiate a partial position and are positively associated with the size of the announced partial position, and the degree of the target’s free cash flow \citep{Akhigbe2007}


    \end{itemize}

\subsection{Corporations}

    \begin{itemize}

        \item Corporation (sample selection) - (Collin-Dufresne, Pierre; Fos, Vyacheslav)

        \item but the runup is even larger if the acquirer is a nonfinancial corporation or a private investor.\citep{Brigida2012}

        \item Third, among 13D filings, the level of informed trading is higher when the filer is a nonfinancial corporation, private investment firm, intends to merge or acquire, or intends to be an activist investor \citep{Brigida2012}

        \item Akhigbe, Martin, and Whyte (2007) show that toeholds acquired by corporate bidders are more likely to result in a full acquisition when compared with all other toehold acquirers.\citep{Brigida2012}

        \item Important table: Target runup by acquirer's identity (t = 0 is the 13D filing date) \citep{Brigida2012}

        \item difference of the paper: Our analysis differs from earlier analyses...\citep{Brigida2012}

        \item financial strength forecasts returns \citep{Choi2012}

        \item Partial bids initiated by corporate bidders are more likely to result in a full acquisition, and the size of the acquired stake and the level of institutional ownership are positively linked to the probability of acquisition. \citep{Akhigbe2007}

        \item Further, studies show that establishing prior ownership increases the bidder’s chance of a successful full acquisition. \citep{Akhigbe2007}

        \item While no study has directly investigated between the link of the partial bidder and the ... \citep{Akhigbe2007}

        \item Without exception, BIDCORP is positive and significant. Partial positions taken by corporate bidders (BIDCORP) generate significantly higher gains to the PTs. This result may reflect the hubris-based view (Roll, 1986) that corporate bidders are likely to overpay in the event of a full takeover. \citep{Akhigbe2007}

    \end{itemize}

\subsection{Difference to prior Research}

    \begin{itemize}

        \item difference of the paper: Our analysis differs from earlier analyses...\citep{Brigida2012}

        \item Focus on the target and not on the investor 

        \item only been seen from the targets perspective: First, does financial strength predict subsequent institutional demand? \citep{Choi2012}

        \item Based on these studies, the economic significance of these partial positions is evident and the link between partial positions and acquisition likelihood is an important issue to examine \citep{Akhigbe2007}
    \end{itemize}

\section{Sample Selection}

    \begin{itemize}
        
        \item The sample of trades by Schedule 13D filers is constructed as follows.\citep{Collin-Dufresne2015}

        \item We compile data from several sources.Stock returns, volume, and prices come from the Center for Research in Security Prices (CRSP). Intraday transactions data (trades and quotes) come from the Trade and Quote (TAQ) database. Data on trades by Schedule 13D filers come from Schedule 13D filings (available on EDGAR) \citep{Collin-Dufresne2015}

        \item We exclude the acquirer’s stated intent within this model, as the set of variables indicating the acquirer type is highly correlated with the set of variables indicating the acquirer’s intent \citep{Brigida2012}

        \item We report that 10-K document file size provides a simple readability proxy that outperforms the Fog Index, does not require document parsing, facilitates replication, and is correlated with alternative readability constructs.\citep{Loughran2014}

        \item Since our focus is on portfolio investments, we restrict our sample by cross-referencing the 13D filings with a list of investment managers that have filed a Schedule 13F holdings report at some point in their history. We do this so as not to confuse corporate crossholdings with activism from portfolio investors. This restriction limits our data somewhat, because only institutions holding more than dollar 100 million in US stocks file 13F reports. \citep{Greenwood2009}
    \end{itemize}

\subsection{Filters}

    \begin{itemize}

    \item Construction of the SC13D Filings Sample - We retain only assets whose CRSP share codes are 10 or 11, that is, we discard certificates, ADRs, shares of beneficial interest, units, companies incorporated outside the United States, Americus Trust components, closed-end funds, preferred stocks, and Real Es- tate Investment Trusts (REITs).\citep{Collin-Dufresne2015}  

    \item Further following Fama and French, we: (1) exclude financials, (2) require firms to have Center for Research in Security Prices (CRSP) share codes 10 or 11 (i.e., ordinary shares), and (3) require firms to have total assets of at least 25 million and book equity of at least 12.5 million. \cite2p{Choi2012}

    \item Brav hedge funds, Collin all, my paper just corporations - The evidence is consistent with Brav et al. (2008) and Klein and Zur (2009), who report a significant positive stock reaction to the announcement of hedge fund activism, where the announcement is triggered by Schedule 13D filings. There are two main differences between our samples. First, we consider all Schedule 13D filings while Brav et al. (2008) and Klein and Zur (2009) consider only filings by hedge funds. Second, a Schedule 13D filing is required to have in- formation on trades in order to be included in our sample. That is, we restrict our sample to cases in which the Schedule 13D filer actively accumulates shares and crosses the 5\% threshold.\citep{Collin-Dufresne2015}

    \item We exclude filings for targets in the financial and utility industries \citep{Brigida2012}

    \item We use company Web sites, newspaper articles, and the Center for International Securities and Derivatives Markets (CISDM) hedge fund database to determine whether or not the activist is a hedge fund or another type of investor (i.e., a mutual fund, pension fund, or investment management company) \citep{Greenwood2009}

    \item Since our focus is on portfolio investments, we restrict our sample by cross-referencing the 13D filings with a list of investment managers that have filed a Schedule 13F holdings report at some point in their history. We do this so as not to confuse corporate crossholdings with activism from portfolio investors. This restriction limits our data somewhat, because only institutions holding more than dollar 100 million in US stocks file 13F reports. \citep{Greenwood2009} 

    \item Outlook on what could have been done to filter more
    \end{itemize}

\section{Summary Statistics for the sample - Investor \& Target}

    \begin{itemize}

        \item Table to show the sample-selection process: Table 1 Sample selection 10-K \citep{You2009}

        \item Characteristics of the Investor/Target (size, key figures) (Coffee Jr., John C.; Palia, Darius)

        \item Distribution of the Filings across the time window

        \item Outlook of what could be done or what will follow: depicts the average daily trading volume (number of shares traded scaled by the number of shares outstanding) over the 21 trading days centering around the 10-K filing date \citep{You2009}

        \item Fig. 1. Abnormal returns around activist filing by outcome \citep[p. 370]{Greenwood2009}

        \item Important table -- Abnormal stock returns surrounding the initial schedule 13D filing dates \citep{Klein2009}

        \item Based on these categories, the vast majority of acquisitions involve targets in the manufacturing sector (48.91\%) followed by services (15.58\%) and financials (10.69\%) \citep{Akhigbe2007}

        \item number of growth/value stocks? 

        \item distribution of the f-scores across targets and investors

    \end{itemize}

\section{Financial Condition -- Investor}

    \begin{itemize}
        \item What do I use? 
        \item Why can I use it?
        \item Control variables - implementation?

        \item only been seen from the targets perspective: First, does financial strength predict subsequent institutional demand? \citep{Choi2012}

        \item expected stock returns are related to three variables: the book-to-
        market equity ratio (Bt/Mt), expected profitability, and expected investment.\citep{Fama2006}
        
        \item I chose nine fundamental signals to measure three areas of the firm's financial condition: profitability, financial leverage/liquidity, and operating efficiency \citep{Piotroski2000}. I define the aggregate signal measure, F-SCORE, as the sum of the nine binary signals. The aggregate signal is designed to measure the overall quality, or strength, of the firm's financial position, and the decision to purchase is ultimately based on the strength of the aggregate signal.

        \item We compute two summary measures of firm strength. The first, OHt, is a mea- sure of bankruptcy risk developed by Ohlson (1980).\citep{Fama2006}

        \item Figure 1 shows the valuation methods most widely used by Morgan Stanley Dean Witter’s analysts for valuing European companies. Surprisingly, the discounted cash flow (DCF) is in fifth place, behind multiples such as the PER, the EV/EBITDA and the EV/EG \citep{Fernandez2001}

        \item One of the most prominent of these fundamental indicators is ‘Z-Score’ (Altman, 1964), which shows statistically significant results in predicting bankruptcy of a company \citep{Mohr2012}

    \end{itemize}

\subsection{F-Score}

    \subsubsection{What is it?}

        \begin{itemize}

            \item Piotroski’s (2000, 2005) f-score is the sum of nine binary signals that form a “…composite measure of firm strength” [Fama and French (2006, page 496)] \citep{Choi2012}

            \item The second composite measure of firm strength, PTt, is from Piotroski (2000).It is the sum of nine binary variables, each equal to 1 if a given condition holds and 0 otherwise.\citep{Fama2006}

            \item Financial Performance Signals used to Differentiate high BM Firms \citep{Piotroski2000}

            \item Instead, F-Score considers a) in what directions the fundamentals of a company are trending and b) whether general financial health conditions are met (i.e. “positive RoA: yes/no”; “equity issuance: yes/no”; “positive accruals yes/no” etc.). F-Score consist of nine binary variables that can be clustered into three dimensions of company health: profitability, balance sheet health and operating efficiency.\citep{Mohr2012}
        
            \item In an effort to apply the general idea of F-Score to growth stocks, Mohanram (2005) constructs an indicator (“G-Score”) that is supposed to better reflect the underlying fundamental situation of growth companies. \citep{Mohr2012}

            \item Mohanram observes the highest predictive ability of G-Score within the largest and most widely followed sub-segment of the growth stock universe. This is an interesting contrast to Piotroski’s (2000) findings that fundamental analysis bears most fruit in a slow information dissecting environment. \citep{Mohr2012}

            \item I refer to companies with a F-Score of 0-3 as “low F-Score” and to companies with a F-Score of 7-9 as “high F- Score”. This is different from Piotroski (2000), as he referred to 0-1 F-Score stocks as “low F-Score” and to 8-9 F-Score stocks as “high F-Score”. I deviate from this approach to arrive at a larger sub-sample and to be independent from rare outliners. \citep{Mohr2012}
        \end{itemize}

    \subsubsection{Why is it used?}

        \begin{itemize}

            \item Our key finding is that, consistent with previous evidence from both developed and emerging market studies, stocks with a high F score earn a significant return premium over stocks with a low F score. \citep{Hyde2014} - justification using the f-score to determine company strength. High fscore leads to higher returns hence stronger firms have higher returns. Result: f-score can be used as a proxy for financial strength in comparison to market return. If the returns are higher for high f-scores, the investors must see a high f-score as a representation of financial strength

            \item We use f-score as the financial strength metric because: (1) it forecasts returns even after accounting for other known stock return predictors such as size, book to market, and asset growth (Fama and French, 2006), (2) the f-score components are commonly used in financial statement analysis, and (3) f-score forecasts profitability consistent with the explanation that f-score proxies for expected profitability (Fama and French, 2006).\citep{Choi2012} - check appendix 

            \item OHt produces strong negative average slopes when used alone to forecast profitability; higher probability of default is (not surprisingly) associated with lower future profitability. But in the multiple regressions, OHt loses most of its explanatory power, at least for forecasts more than a year ahead. In contrast, though the
            positive average slopes on the PTt measure of firm strength are smaller when other variables are in the profitability regressions, they remain more than 2.3 standard errors
            from zero \citep{Fama2006}

            \item The Piotroski (2000) and Ohlson (1980) measures of firm strength, which are proxies for expected net cash flows (earnings minus investment), are also related to average returns in the manner predicted by Eq. (3) \citep{Fama2006}

            \item financial variables that reflect changes in these economic conditions should be useful in predicting future firm performance. This logic is used to identify the financial statement signals incorporated in this paper. \citep{Piotroski2000}

            \item Importance f-score: relative to broader variables capturing changes in the overall health of these companies \citep{Piotroski2000}

            \item This approach represents one simple application of fundamental analysis for identifying strong and weak value firms \citep{Piotroski2000}

            \item This finding confirms earlier research by Piotroski (2004) who states that F-Score does not lose its predictive ability when applied to growth (instead of value) stocks. \citep{Mohr2012}

        \end{itemize}

    \subsubsection{Problems/Comments}

        \begin{itemize}

            \item Outlook - Capital structure stability is the exception, not the rule \citep{Deangelo2015}

            \item but none has permanently maintained even approximately stable leverage \citep{Deangelo2015}

            \item The evaporating similarity of cross-sections raises questions about the empirical relevance of leverage targeting \citep{Deangelo2015}

            \item What we do know is that the targets of hedge fund activism are not randomly distributed, but rather tend to have some common characteristics, including in most (but not all) studies a low Tobin’s Q, below average leverage, a low dividend payout, and a “value,” as opposed to a “growth,” orientation. \citep{CoffeeJr.2014}


            \item See Fama and French (2006) for evidence that accruals proxy for future profitability and forecast returns, Haugen and Baker (1996) for evidence ROE proxies for future profitability and forecasts returns and Fama and French (2006) and Chen, Novy-Marx, and Zhang (2011) for evidence ROA proxies for future profitability and forecasts return \citep{Choi2012}
            
            \item Jensen (2004) argues that managers are more inclined to apply aggressive accounting when their companies’ stock price is valuated excessively \citep{Mohr2012}
        \end{itemize}


\subsection{Expected Return}    

    \begin{itemize}

        \item Based on accounting fundamentals!! 

        \item T

        \item The base accounting variables, from Compustat, are At, total assets (Compustat data item 6); Yt, income before extraordinary items (18); ACt, accruals [the change in current assets (4), minus the change in cash and short term investments (1), minus the change in current liabilities (5), plus the change in debt in current liabilities (34)]; Dt, total dividends [dividends per share by ex date (26) times common shares outstanding (25)]; and Bt, book equity  [total assets (6), minus liabilities (181), plus balance sheet deferred taxes and investment tax credit (35) if available, minus preferred stock liquidating value (10) if available,or redemption value (56) if available, or carrying value (130)]. The accounting variables for year t are measured at the fiscal year end that falls in calendar year t. \citep{Fama2006}

        \item We include firm size (the log of total market cap, lnMCt) among the fundamental variables because smaller firms tend to be less profitable (Fama and French, 1995).\citep{Fama2006}

        \item Expected returns increase in profitability and decrease in accruals. We show that cash-based operating profitability (a measure that excludes accruals) outperforms measures of profitability that include accruals \citep{Ball2016}

        \item In our analyses, any increase in profitability that is solely due to accruals themselves has no relation with the cross section of returns.\citep{Ball2016}

        \item In other words, the evidence implies that only the cash-based component of operating profits matters in the cross section of expected returns, and the predictive power of accruals is attributable to their negative correlation with the cash-based component \citep{Ball2016}
       
        \item Using a multivariate regression, we test whether FDR still has significant correlation with post-filing stock returns after controlling for the standardized unexpected earnings (SUE)in earnings release \citep{You2009}

    \end{itemize}

\subsection{Cross-Sectional Regression}

    \begin{itemize}
        \item To do so, I build a multifactor regression that consists of the explanatory factors size, P/B, momentum, accruals, equity offerings and F-Score. My model closely matches the model used by Piotroski (2000, p. 22) \citep{Mohr2012}

        \item The factors in this regression are based on widely-quoted research. Indeed, size effect and P/B are components of the original three-factor model (Fama and French, 1992) \citep{Mohr2012}

        \item Cross-sectional regression results \citep[p.3094]{Akhigbe2007}
    \end{itemize}


\subsection{Measurements (variables) for the financial condition} 
    
    \begin{itemize}

        \item industry multiples
        \item Balance sheet ratios
        \item ROE, ROA, accruals 
        \item Fundamental analysis
        \item Working capital adequacy
        \item Asset performance
        \item Capitalization structure

    \end{itemize}
    
\subsection{Evaluation of the financial strength} 
    
    \begin{itemize}

        \item Interpretation of the variables
        \item Group investors based on the properties
        \item Pitroskis'f-score as a proxy for financial strength
        \item Use the expected returns of the investors as a proxy for company strength???? If so, the components to measure the expected return can be used to form different groups of investors. Another possibility could be to calculate the expected returns and then form groups of investors? 
        \item Average return over the last xx years 

    \end{itemize}

\section{Financial Condition -- Target}

    \begin{itemize}

        \item Who are the targets - many report that the typical target firm of an activist investor is smaller, more profitable, has a large institutional ownership level, and has more of a “value” orientation (namely a higher book to market ratio) \citep{CoffeeJr.2014}

        \item the evidence consistently supports only the generalization that targets of activism often tend to have a lower Tobin’s Q and a “value” orientation \citep{CoffeeJr.2014}

        \item For example, Brav, Jiang, Partnoy and Thomas, supra note 8, find no statistically significant relationship between the target’s abnormal returns and their governance and capital structure \citep{CoffeeJr.2014}

        \item These findings are consistent with the idea that the F Score is most effective when applied to stocks for which the market is slow to incorporate relevant financial information. Deep value stocks are typically neglected by analysts and investors and thus likely to exhibit slow impounding of new information \citep{Hyde2014}

        \item the ‘slow impounding of new information’ hypothesis by showing that future institutional investor demand is high for stocks with high F scores. \citep{Hyde2014}

        \item The finding by Mohr (2012) that the F score effectively discriminates between high and low return stocks amongst growth stocks provides additional justification for broadening the analysis to include all stocks. \citep{Hyde2014}

        \item Our key finding is that, consistent with previous evidence from both developed and emerging market studies, stocks with a high F score earn a significant return premium over stocks with a low F score.\citep{Hyde2014}

        \item First, does financial strength predict subsequent institutional demand? \citep{Choi2012}

        \item Analogue to investors? Specifically, the difference between high and low f-score group returns averages 25.73\% (statistically significant at the 1\% level). \citep{Choi2012}

        \item Consistent with Piotroski (2000, 2005) and Fama and French (2006), the results reveal a strong positive relation between f-score 10 and future returns. High f-score stocks average annual market-adjusted returns 8.35\% greater (statistically significant at the 1\% level) than low f-score stocks \citep{Choi2012} 

       \item One of the most prominent of these fundamental indicators is ‘Z-Score’ (Altman, 1964), which shows statistically significant results in predicting bankruptcy of a company \citep{Mohr2012}

       \item Thus, whether we examine stock returns or accounting data, we conclude that hedge fund activists, on average, target better-performing firms than do other entrepreneurial activists \citep{Klein2009}

       \item These findings make an interesting comparison to those reported by Bethel et al. (1998), who find that between 1980 and 1989, activist investors were more likely to purchase large blocks of shares in firms with relatively low EBITDA/assets \citep{Klein2009}

       \item Table II supports the view that targets of hedge funds have substantially more cash than do other entrepreneurial activist targets on their balance sheets, be it cash or cash plus investments \citep{Klein2009}

       \item The likelihood of acquisition decreases with firm size \citep{Akhigbe2007} -- for further literature check p. 3084 onwards.
    \end{itemize}


\section{Justification for using the underlying Inputs to determine the Financial Condition}

    \begin{itemize}

        \item Information/Data Availability to the Market
        
        \item To ensure that investors would have the necessary information to compute f-scores, Piotroski (2000, 2005) examines annual returns beginning the fifth month following fiscal year end. \citep{Choi2012}
        
        \item Check back with sample selection: We report that 10-K document file size provides a simple readability proxy that outperforms the Fog Index, does not require document parsing, facilitates replication, and is correlated with alternative readability constructs.\citep{Loughran2014}

        \item Readability: financial information is definitely available - is it readable? (Loughran, MacDonald)
        
        \item Because information is incorporated in a given manner, we can use several variables as proxies for financial strength observed by the market in the form of abnormal returns (they know what we know)
        
        \item At the time a company files its 10-K report with the SEC, most likely all key information has already been disclosed to the public.\citep{You2009}
        
        \item 10-K Filings - MANAGERS OF PUBLICLY traded firms are required to produce public documents that provide a comprehensive review of the firm’s business operations and financial condition. An important financial disclosure document created by managers to communicate with investors and analysts is the annual report filed pursuant to the Securities Exchange Act of 1934, Form 10-K.\citep{Loughran2014}

        \item Second, we recommend using the file size of the 10-K as an easily calculated proxy for document readability \citep{Loughran2014}

        \item Check with Company Condition: Figure 1 shows the valuation methods most widely used by Morgan Stanley Dean Witter’s analysts for valuing European companies. Surprisingly, the discounted cash flow (DCF) is in fifth place, behind multiples such as the PER, the EV/EBITDA and the EV/EG \citep{Fernandez2001}
        
    \end{itemize}

\section{Abnormal Returns - Event Study} 


    \begin{itemize}
        
        \item ECONOMISTS are frequently asked to measure the effects of an economic event on the value of firms \citep{MacKinlay1997}

        \item Short-horizon event studies of stock returns: Many studies have examined what happens to targets firm’s stock price when there is a Schedule 13D filing with the SEC \citep{CoffeeJr.2014}

        \item Consistency with prior findings/literature

        \item Trading strategy of Schedule 13D filers before the filing day \citep{Collin-Dufresne2015}
        
        \item In the majority of applications, the focus is the effect of an event on the price of a particular class of securities of the firm, most often common equity. \citep{MacKinlay1997}

        \item Useful papers which deal with the practical importance of many of the complications and adjustments are the work by Stephen Brown and Jerold Warner published in 1980 and 1985. The 1980 paper considers implementation issues for data sampled at a monthly interval and the 1985 paper deals with issues for daily data \citep{MacKinlay1997}

        \item I consistently use the mid-price between ask and bid for a stock at the specific date t0 and t1. \citep{Mohr2012}

        \item Ex ante abnormal return generating models \citep{Kolari2010}

        \item Specifically, hedge fund targets earn 10.2\% average abnormal stock returns during the period surrounding the initial Schedule 13D. Other activist targets experience a significantly positive average abnormal return of 5.1\% around the SEC filing window \citep{Klein2009}

        \item Describe the mathematics/procedure behind it! Formulas! 

        \item Power functions to determine the power of the analysis

    \end{itemize}

\subsection{Windows}

    \begin{itemize}

        \item Figure 2 plots the average buy-and-hold return, in excess of the buy-and-hold return on the value-weightedNYSE/Amex/NASDAQ index from CRSP, from 60 days prior to the filing date to 40 days afterward.\citep{Collin-Dufresne2015}

        \item For ease of exposition, let us define [-x, +y] to be x days before the 13D filing, to y days after the filing. On this basis - literature \citep{CoffeeJr.2014}

        \item event-window: we find that most informed trading before a 13D filing is during the event window (-10, -6).\citep{Brigida2012}


        \item event window - We also found that the target runup before a 13D filing is greatest during the event window (-10,-6). Therefore, future academic research that estimates the share price response surrounding 13D filings should use a window extending to at least 10 days prior to the filing. \citep{Brigida2012}

        \item Runup = cumulative abnormal return of the target’s stock over the intervals (-10, -1), (-5, -1), and (-2, -1) relative to t = 0 being the filing of the Schedule 13D or 13G. \citep{Brigida2012}
        
        \item In fact, 92\% of the effect of a Schedule 13D filing on the target’s stock is realized before 3 days prior to the filing. These results show very little new information is revealed to the market when the Schedule 13D filing is made public \citep{Brigida2012}

        \item Therefore, any analysis of the effect of a Schedule 13D filing on the target stock should consider an event window starting no later than ten days before the filing. \citep{Brigida2012}

        \item Our event window begins on day –30 to allow for the 10-day 13D filing window, possible prior leakage of information, and prefiling price pressure that may occur due to the activist accruing a large stake in a relatively short period of time. We extend the event window to day +5, and alternatively to day +30 to accommodate subsequent press coverage of the filing event \citep{Klein2009}
    \end{itemize}

\subsection{Problems}

    \begin{itemize}

        \item It is well known that event studies are prone to cross-sectional correlation among abnormal returns when the event day is the same for sample firms. \citep{Kolari2010}

        \item When there is event-date clustering, we find that even relatively low cross-correlation among abnormal returns is serious in terms of over-rejecting the null hypothesis of zero average abnormal returns. \citep{Kolari2010}

        \item There have been several other attempts in the literature to resolve the contemporaneous correlation problem (Kothari and Warner 2007).\citep{Kolari2010}

        \item In this article, we have demonstrated that even relatively low cross-sectional correlation in an event study with clustered event days can cause serious over- rejection of the null hypothesis of no event mean effect. \citep{Kolari2010}

    \end{itemize}

\subsection{Abnormal Returns}

    \begin{itemize}

        \item We use the buy-and-hold method to measure the abnormal stock returns for two reasons. (1) As shown in Barber and Lyon (1997), buy-and-hold is favored over cumulative abnormal return (CAR) on a conceptual ground. (2) BHAR facilitates the cross-sectional analysis of how abnormal return varies with complexity. However, as pointed out by Mitchell and Stafford (2000), BHAR may exaggerate the short-term abnormal return due to compounding. To address this issue, we conduct a calendar time analysis. Specifically, each month we place firms into five portfolios based on their most recent FDR \citep{You2009}

        \item We find that the market reacts favorably to activism, consistent with the view that it creates value. The filing of a Schedule 13D revealing an activist fund’s investment in a target firm results in large positive average abnormal returns, in the range
        of 7\% to 8\%, during the (–20,+20) announcement window \citep{Brav2008}

        \item We define market-adjusted returns as the firm’s buy and hold return less the CRSP value-weighted index buy and hold return over the same period. \citep{Choi2012}

        \item Also one can easily modify the statistical framework so that the analysis of the abnormal returns is autocorrelation and heteroskedasticity consistent by using a generalized method-of-moments approach.\citep{MacKinlay1997}

        \item The market model represents a potential improvement over the constant mean model \citep{MacKinlay1997}

        \item Restrictions of the CAPM: The use of the Capital Asset Pricing Model is common in event studies of the 1970s \citep{MacKinlay1997} 

        \item To reduce any effect of outliers on the estimated coefficients, we also estimate each equation using robust regression employing the Huber weight function. \citep{Brigida2012}

        \item The target’s size- adjusted return is the difference between its buy-and-hold return over a se- lected time period and the buy-and-hold return for the same time period on the Fama–French size-matched portfolio of firms. 
        
        \item The market-adjusted return is the difference between the target’s buy-and-hold return and the value-weighted NYSE/Amex/Nasdaq index from CRSP. 
        
        \item The industry-adjusted return is the difference between the target’s buy-and-hold return and the return for all firms (target excluded) in the target’s Fama–French (1997) 48-industry code. \citep{Klein2009}
    
        \item In summary, Table IV shows that the market reacts positively to activism in general and that the positive abnormal returns are robust across different methodologies. \citep{Klein2009}

        \item We show that these returns are largely explained by the ability of activists to force target firms into a takeover \citep{Greenwood2009}

        \item We estimate mean cumulative abnormal returns (CARs) for various intervals surrounding the announcement: (?11,?2), (?1,0) (?1,+1) and (+2,+10). \citep{Akhigbe2007}


    \end{itemize}

\subsection{Statistical Tests}

    \begin{itemize}

        \item computed from the time-series of the differences in the 24 cross-sectional means with Newey-West (1987) standard errors] \citep{Choi2012}

        \item The buy-and-hold Benchmark approach: The first approach uses a benchmark to measure the abnormal buy-and-hold return for every event firm, and tests the null hypothesis that the average abnormal return is zero.\citep{ang2011}

        \item Testing the error terms from ordinary least-squares estimations of the below regression equations,using the Breusch-Pagan test, found no evidence for significant heteroscedasticity \citep{Brigida2012}

        \item Also, when testing cumulative abnormal returns (CARs) in multiple-day windows, our test statistic increasingly dominates nonparametric tests as the window is lengthened \citep{Kolari2010}

        \item For example, theWilcoxon (1945) rank-sum test has relatively higher power com- pared with parametric tests, particularly for fat-tailed distributions \citep{Kolari2010}

        \item Corrado (1989) and Corrado and Zivney (1992) recommend non- parametric rank and sign tests that are expected to be robust against event-induced volatility and cross-correlation. The most popular approach for testing CARs with these methods is a cumulated ranks test. Over a small num- ber of periods, this cumulative rank test is able to detect abnormal behavior (Cowan 1992; Campbell andWasley 1993, 1996).\citep{Kolari2010}

        \item Particularly relevant to the present study, parametric tests based on scaled abnormal returns methods have been found to be superior in terms of power over those based on non-scaled returns \citep{Kolari2010}

        \item The most widely used scaled tests are the t-statistics of Patell (1976) and Boehmer, Musumeci, and Poulsen (1991). \citep{Kolari2010}

        \item Thus, scaled returns should be used only for statistical testing purposes as signal detection devices of the event effect, while raw returns carry the economic information for inter- pretation purposes when a signal is detected \citep{Kolari2010}

        \item For further reading of test statistics -- Other test statistics \citep{Kolari2010}

        \item Table IV presents abnormal stock returns and both parametric and non- parametric test statistics to evaluate whether these returns are different from zero. \citep{Klein2009}
    \end{itemize}

\section{Investor-Target Ratio} 

    \begin{itemize}

        \item These findings are consistent with the idea that the F Score is most effective when applied to stocks for which the market is slow to incorporate relevant financial information. Deep value stocks are typically neglected by analysts and investors and thus likely to exhibit slow impounding of new information \citep{Hyde2014}
        
        \item What is the relation between the investor-target ratio and the market?
        
        \item Reference to the financial condition of the target: What are the characteristics?
        
        \item What kind of companies are the targets (Coffee Jr., John C. Palia, Darius)

        \item Vertical integration, scale effect - check with industry code - goal takeover?! 
        

        % \item useful signals regarding the likelihood of acquisition (Walkling, 1985, and Akhigbe, Martin, and Whyte, 2007) \citep{Brigida2012}

        % \item Since a 13D filing can influence the likelihood that a firm will become a takeover target, it is not surprising that the filing elicits a market reaction.\citep{Brigida2012}

    \end{itemize}

\section{Data}

    \subsection{COMPUSTAT - Financial Condition}
    
     \subsubsection{Screening Variables}

        \begin{enumerate}
            \item What consolidation level? - Consolidated
            \item What industry? No financial services (FS) 
            \item What data format?  - Standardized 
            \item Population source? - Domestic 
            \item Currency? - USD 
            \item Company Status? - Active \& Inactive  
        \end{enumerate}

    \subsubsection{Variables}

        \begin{enumerate}
        \item Identifying Information

            \begin{itemize}
                \item Company name 
                \item CIK number 
            \end{itemize}
    
        \item Identifying Information cont.  
        
            \begin{itemize}
                \item GIC variables - GIC sectors etc.
                \item NAICS - in addition to GIC? 
                \item SIC - in addition to GIC?
            \end{itemize}

        \item Company Descriptor

            \begin{itemize}
                \item Acquisition method? - ACQMETH filter by takeover? 
                \item 
            \end{itemize}
        
        \item Balance Sheet Items 

            \begin{itemize}
                \item Current Assets total (ACT)
                \item Total Assets (AT)
                \item Account receivables total (ARTFS)
                \item Cash (CH)
                \item Liabilities total (LT) 
                \item Long term debt total (DLTT)
            \end{itemize}
        
        \subsection{F-Score Variables}

            \begin{enumerate}
                \item Positive net income before extraordinary items -- IB
                \item Positive cash flow from operations

                    \begin{enumerate}

                      \item If a company files a statement of working capital (Format Code 1)

                      \begin{enumerate}
                        \item \emph{cash flow from operations} is \emph{funds fromm operations} less \emph{other changes in working capital WCAPC}.
                        \item \emph{Funds from operations} is the sum of \emph{earnings before extraordinary items IB} \emph{income statement deferred taxes TXDI} and \emph{equity's share of depreciation expenses}.
                        \item \emph{equity's share of depreciation expenses} is \emph{depreciations expense DP} times \emph{the ratio of market capitalization} to the sum of \emph{market capitalization} and the difference between \emph{total assets AT} and \emph{book value of equity}.
                        \item \emph{book value of equity} is defined as \emph{total assets AT} less \emph{liabilities LT} plus \emph{deferred taxes and investment tax credits TXDITC} less \emph{preferred stocks liquidity value PSTKL} or \emph{preferred stock redemption value PSTKRV} or \emph{preferred stocks carrying value PSTK}
                      \end{enumerate}
                      
                      \item If a company files a statement of cash flows (Format code 7) 

                      \begin{enumerate}
                          \item \emph{cash flow from operations} is \emph{net cash flow from operating activities OANCF}
                      \end{enumerate}

                      \item For all other Compustat format codes

                      \begin{enumerate}
                          \item \emph{cash flow from operations} is the sum of \emph{funds from operations} and \emph{changes in working capital WCAPC}
                      \end{enumerate}

                    \end{enumerate}  

                \item \emph{Cash flow from operations} greater than \emph{net income} -- (2) larger than (1)

                \item Growth in net income (scaled by total assets) from the prior fiscal year end 

                    \begin{enumerate}
                        \item \emph{net income before extraordinary items IB} divided by \emph{total assets AT} 
                    \end{enumerate}

                \item Decrease in leverage from prior fiscal year end 

                    \begin{enumerate}
                        \item \emph{long term debt DLTT + DD1} divided by \emph{total assets AT} 
                    \end{enumerate}
                    
                \item Increase in liquidity from prior fiscal year end 

                    \begin{enumerate}
                        \item Ratio of \emph{current assets ACT} to \emph{current liabilities LCT}
                    \end{enumerate}
                
                \item No new common or preferred stock issued over the previous year 

                    \begin{enumerate}
                        \item If sales from \emph{common and preferred stocks is zero SSTK}
                    \end{enumerate}

                \item Increase in gross margin from prior fiscal year end 

                    \begin{enumerate}
                        \item One minus \emph{ratio of costs of goods sold COGS} to \emph{sales SALE}
                    \end{enumerate}

                \item Increase in asset turnover prior to fiscal year end 

                    \begin{enumerate}
                        \item \emph{ratio of sales SALE} to \emph{total assets at the beginning of the year AT -- AT from prior fiscal year}
                    \end{enumerate}


            \end{enumerate}
                    
            

    \end{enumerate}


    \subsection{CRSP - Event Study Abnormal Returns}


\section{Literature}

    % \begin{itemize}

    %     \item Kothari, S. P., and J. B. Warner. 2007. Econometrics of Event Studies. In B. E. Eckbo (ed.), Handbooks of Corporate Finance: Empirical Corporate Finance, Chapter 1. Amsterdam: Elsevier/North-Holland.

    %     \item Journal of Financial and Quantitative Analysis (JFQA)

    %     \item Fama, E. F., and K. R. French. 2006. Profitability, Investment and Average Returns. Journal of Financial Economics 82: 491-518. (page 496)

    %     \item Mohr, J-H.M., 2012. Utility of Piotroski's F-score for predicting growth stock returns. Working paper, MFIE Capital.

    %     \item The econometric of Financial Markets - Campbell, Lo and MacKinaly 1997

    %     \item signs for takeover? 

    % \end{itemize}

\printbibliography

\end{document}