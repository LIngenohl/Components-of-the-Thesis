\documentclass[12pt]{article}
	\usepackage[T1]{fontenc}
	\usepackage[utf8]{inputenc}
	\usepackage[british]{babel}
	\usepackage[a4paper]{geometry}
	\geometry{verbose,tmargin=3cm,bmargin=3.5cm,lmargin=4cm,rmargin=3cm,marginparwidth=70pt}
	\setcounter{secnumdepth}{3}
	\setcounter{tocdepth}{3}
	\usepackage{prettyref}
	\usepackage{textcomp}
	\usepackage{setspace}
	\usepackage{indentfirst}
	\usepackage{fancyhdr}
	\usepackage{url}
	\usepackage[normalem]{ulem}
	\usepackage[table, fixpdftex]{xcolor}
	\usepackage{algpseudocode}
	\usepackage{bigstrut}
	\usepackage{enumitem}

	% package hyperref
    \usepackage{hyperref}
    
	% biblatex
	\usepackage[style=authoryear-icomp,natbib=true,maxcitenames=2, maxbibnames=11,backend=biber,pagetracker=page,hyperref=true]{biblatex} \usepackage{csquotes}
	\renewcommand*{\bibsetup}{%
		\interlinepenalty=10000\relax % default is 5000
		\widowpenalty=10000\relax
		\clubpenalty=10000\relax
		\raggedbottom
		\frenchspacing
        \biburlsetup}
        
	% fixes the page number of the first page of each chapter
	\fancypagestyle{plain}{
			\fancyhead{}
			\renewcommand{\headrulewidth}{0pt}
			\renewcommand{\footrulewidth}{0pt}
			\fancyfoot[OC]{\begin{flushright}\thepage\end{flushright}}
    }
    
	% fancy headers for the thesis
	\fancyhead{}
	\fancyhead[RO]{\slshape \nouppercase \rightmark}
	\fancyfoot[OC]{\begin{flushright}\thepage\end{flushright}}
	\renewcommand{\headrulewidth}{0.4pt}
	\setlength{\headheight}{14pt}

	% add bibliography database
	\addbibresource{BA Kopie.bib}
	
	% space between biblio items
	\setlength\bibitemsep{1.7\itemsep} 
	
	% title without ""
	\DeclareFieldFormat[inbook]{title}{#1}
	% non-italic
	\DeclareFieldFormat[online]{tlaitle}{#1} 
	% title unquoted
	\DeclareFieldFormat[article]{title}{#1} 
	% no pp. 
	\DeclareFieldFormat[article]{pages}{#1} 
	% bold volume
	\DeclareFieldFormat*{volume}{\mkbibbold{#1}\setpunctfont{\textbf}}
	
	% no in:
	\renewbibmacro{in:}{} 
	
	% (volume)
	\renewbibmacro*{volume+number+eid}{%
			\printfield{volume}%
			%\setunit*{\adddot}% DELETED
			% \setunit*{\addnbspace}% NEW (optional); there's also \addnbthinspace
			\printfield{number}%
			% \setunit{\addcomma\space}%
			\printfield{eid}}
	\DeclareFieldFormat[article]{number}{\mkbibparens{#1}} 
	
	% edition.
	\DeclareFieldFormat{edition}%
	{(\ifinteger{#1}%
			{\mkbibordedition{#1}\addthinspace{}ed.}%
			{#1\isdot}).}
	
	% publisher and location position
	\renewbibmacro*{publisher+location+date}{%
			\printlist{publisher}%
			\setunit*{\addcomma\space}%
			\printlist{location}%
			\setunit*{\addcomma\space}%
			\usebibmacro{date}%
			\newunit}
	
	% shortauthor before author
	\renewbibmacro*{begentry}{%
			\ifkeyword{Key}{\sffamily}{}%
			\iffieldundef{shorthand}
			{}
			{\global\undef\bbx@lasthash
					\printfield{shorthand}%
					\addcolon\space}%
			\ifboolexpr{test {\usebibmacro{bbx:dashcheck}} or test {\ifnameundef{shortauthor}}}%
			{}%
			{\printnames{shortauthor}%
                    \addspace\textendash\space}}
                    
\title{Components of the Thesis}
\author{Leopold Ingenohl}


\begin{document}
\maketitle


\section{Motivation \& Importance} 

    \begin{itemize}
        \item Investor activism 
        \item Corporation (sample selection) - (Collin-Dufresne, Pierre; Fos, Vyacheslav)
        \item 13D Filings (Collin-Dufresne, Pierre; Fos, Vyacheslav)
        \item An average Schedule 13D filing in our sample is characterized by a positive and significant market reaction upon announcement \citet{Collin-Dufresne2015}
        \item these activist shareholders know they can increase the value of the firm they invest in by their own effort (e.g., shareholder activism).\citet{Collin-Dufresne2015}
        \item Importance - who are the targets of activism? \citet{CoffeeJr.2014}
        \item Short-horizon event studies of stock returns: Many studies have examined what happens to targets firm’s stock price when there is a Schedule 13D filing with the SEC \citet{CoffeeJr.2014}
        \item Third, among 13D filings, the level of informed trading is higher when the filer is a nonfinancial corporation, private investment firm, intends to merge or acquire, or intends to be an activist investor \citet{Brigida2012}
        \item Brav, Jiang, Partnoy, and Thomas (2008) have documented a positive and significant average abnormal return in response to 13D filings \citet{Brigida2012}
        \item difference of the paper: Our analysis differs from earlier analyses...\citet{Brigida2012}
        \item but the runup is even larger if the acquirer is a nonfinancial corporation or a private investor.\citet{Brigida2012}
        \item Within the sample of 13D filings, some of the acquirers are corporations that are potential full-
        acquirers, while other acquirers are institutional investors that are not likely to pursue a complete takeover. \citet{Brigida2012}
        \item Akhigbe, Martin, and Whyte (2007) show that toeholds acquired by corporate bidders are more likely to result in a full acquisition when compared with all other toehold acquirers.\citet{Brigida2012}
        \item Important table: Target runup by acquirer's identity (t = 0 is the 13D filing date) \citet{Brigida2012}
        \item We find that the market reacts favorably to activism, consistent with the view that it creates value. The filing of a Schedule 13D revealing an activist fund’s investment in a target firm results in large positive average abnormal returns, in the range
        of 7\% to 8\%, during the (–20,+20) announcement window \citet{Brav2008}
        \item ECONOMISTS are frequently asked to measure the effects of an economic event on the value of firms \citet{MacKinlay1997}

    \end{itemize}

\section{Filings}

\begin{itemize}
    \item Schedule 13D filings must be made within 10 days of acquiring a beneficial ownership of 5\% or greater of the outstanding common stock of a U.S. public company. The use of the qualifier ‘beneficial’ is important because related, yet different entities, may have to file a schedule 13D if their combined ownership of the target is 5\% or greater and their voting or investment power is combined \citet{Brigida2012}
    \item Filing a Schedule 13D allows the investor to behave in an active manner. \citet{Brigida2012}
    \item Within the Schedule 13D and 13G filings is information important to this analysis. \citet{Brigida2012}
    \item 10K-Finlings - While abnormal trading volume and return volatility may indicate market reaction to new information, they could also simply reflect an increase in noise trading. \citet{You2009}
\end{itemize}


\section{Descriptive Introduction}

    \begin{itemize}
        \item HNA - Deutsche Bank
    \end{itemize}

\section{Sample Selection}

    \begin{itemize}
        \item We compile data from several sources.Stock returns, volume, and prices come from the Center for Research in Security Prices (CRSP). Intraday transactions data (trades and quotes) come from the Trade and Quote (TAQ) database. Data on trades by Schedule 13D filers come from Schedule 13D filings (available on EDGAR) \citet{Collin-Dufresne2015}
        \item The sample of trades by Schedule 13D filers is constructed as follows.\citet{Collin-Dufresne2015}
        \item Construction of the SC13D Filings Sample - We retain only assets whose CRSP share codes are 10 or 11, that is, we discard certificates, ADRs, shares of beneficial interest, units, companies incorporated outside the United States, Americus Trust components, closed-end funds, preferred stocks, and Real Es- tate Investment Trusts (REITs).\citet{Collin-Dufresne2015}
        \item Brav hedgefunds, Collin all, my paper just corporations - The evidence is consistent with Brav et al. (2008) and Klein and Zur (2009), who report a significant positive stock reaction to the announcement of hedge fund activism, where the announcement is triggered by Schedule 13D filings. There are two main differences between our samples. First, we consider all Schedule 13D filings while Brav et al. (2008) and Klein and Zur (2009) consider only filings by hedge funds. Second, a Schedule 13D filing is required to have in- formation on trades in order to be included in our sample. That is, we restrict our sample to cases in which the Schedule 13D filer actively accumulates shares and crosses the 5\% threshold.\citet{Collin-Dufresne2015}
        \item We exclude the acquirer’s stated intent within this model, as the set of variables indicating the acquirer type is highly correlated with the set of variables indicating the acquirer’s intent \citet{Brigida2012}
        \item We exclude filings for targets in the financial and utility industries \citet{Brigida2012}
        \item Further following Fama and French, we: (1) exclude financials, (2) require firms to have Center for Research in Security Prices (CRSP) share codes 10 or 11 (i.e., ordinary shares), and (3) require firms to have total assets of at least 25 million and book equity of at least 12.5 million. \citet{Choi2012}
        \item Sample selection criteria
        \item CRSP share code filter - what shares (Collin-Dufresne, Pierre; Fos, Vyacheslav)
        \item Outlook on what could have been done to filter more asd
    \end{itemize}

\section{Summary Statistics for the sample - Investor \& Target}

\begin{itemize}
    \item Table 1 Sample selection 10-K \citet{You2009}
    \item Characteristics of the Investor/Target (size, key figures) (Coffee Jr., John C.; Palia, Darius)
    \item Distribution of the Filings
    \item depicts the average daily trading volume (number of shares traded scaled by the number of shares outstanding) over the 21 trading days centering around the 10-K filing date \citet{You2009}
\end{itemize}

\section{Financial Condition}

\subsection{Investor}

    \begin{itemize}
        \item expected returns increase in profitability and decrease in accruals. We show that cash-based operating profitability (a measure that excludes accruals) outperforms measures of profitability that include accruals \citet{Ball2016}
        \item In our analyses, any increase in prof- itability that is solely due to accruals themselves has no relation with the cross section of returns.\citet{Ball2016}
        \item In other words, the evidence implies that only the cash-based component of operating profits matters in the cross section of ex- pected returns, and the predictive power of accruals is at- tributable to their negative correlation with the cash-based component \citet{Ball2016}
        \item Outlook - Capital structure stability is the exception, not the rule \citet{Deangelo2015}
        \item but none has permanently maintained even approximately stable leverage \citet{Deangelo2015}
        \item The evaporating similarity of cross-sections raises questions about the empirical relevance of leverage targeting \citet{Deangelo2015}
        \item Control variables - implementation?
        \item What we do know is that the targets of hedge fund activism are not randomly distributed, but rather tend to have some common characteristics, including in most (but not all) studies a low Tobin’s Q, below average leverage, a low dividend payout, and a “value,” as opposed to a “growth,” orientation. \citet{CoffeeJr.2014}
        \item Our key finding is that, consistent with previous evidence from both developed and emerging market studies, stocks with a high F score earn a significant return premium over stocks with a low F score. \citet{Hyde2014} - justification using the f-score to determine company strength. High fscore leads to higher returns hence stronger firms have higher returns. Result: f-score can be used as a proxy for financial strength in comparison to market return. If the returns are higher for high f-scores, the investors must see a high f-score as a representation of financial strength
        \item See Fama and French (2006) for evidence that accruals proxy for future profitability and forecast returns, Haugen and Baker (1996) for evidence ROE proxies for future profitability and forecasts returns and Fama and French (2006) and Chen, Novy-Marx, and Zhang (2011) for evidence ROA proxies for future profitability and forecasts return \citet{Choi2012}
        \item only been seen from the targets perspective: First, does financial strength predict
        subsequent institutional demand? \citet{Choi2012}
        \item financial strength forecasts returns \citet{Choi2012}
        \item iotroski’s (2000, 2005) f-score is the sum of nine binary signals that form a “…composite measure of firm strength” [Fama and French (2006, page 496)] \citet{Choi2012}
        \item We use f-score as the financial strength metric because: (1) it forecasts returns even after accounting for other known stock return predictors such as size, book to market, and asset growth (Fama and French, 2006), (2) the f-score components are commonly used in financial statement analysis, and (3) f-score forecasts profitability consistent with the explanation that f-score proxies for expected profitability (Fama and French, 2006).\citet{Choi2012} - check appendix 
        \item Using a multivariate regression, we test whether FDR still has significant correlation with post-filing stock returns after controlling for the standardized unexpected earnings (SUE)in earnings release \citet{You2009}
        \item expected stock returns are related to three variables: the book-to-
        market equity ratio (Bt/Mt), expected profitability, and expected investment.\citet{Fama2006}
        \item The accounting fundamentals used as explanatory variables in the proxies for expected profitability and investment include lagged values of Bt/Mt, a dummy variable for negative earnings, profitability (Yt/Bt) for firms with positive earnings, accruals relative to book
        equity for firms with positive (+ACt/Bt) and negative (?ACt/Bt) accruals, investment (dAt/At?1), a dummy variable for firms that do not pay dividends (No Dt), and the ratio of dividends to book equity (Dt/Bt).\citet{Fama2006}
        \item We include firm size (the log of total market cap, lnMCt) among the fundamental variables because smaller firms tend to be less profitable (Fama and French, 1995).\citet{Fama2006}
        \item OHt produces strong negative average slopes when used alone to forecast profitability; higher probability of default is (not surprisingly) associated
        with lower future profitability. But in the multiple regressions, OHt loses most of its explanatory power, at least for forecasts more than a year ahead. In contrast, though the
        positive average slopes on the PTt measure of firm strength are smaller when other variables are in the profitability regressions, they remain more than 2.3 standard errors
        from zero \citet{Fama2006}
        \item The Piotroski (2000) and Ohlson (1980) measures of firm strength, which are proxies for expected net cash flows (earnings minus investment), are also related to average returns in the manner predicted by Eq. (3) \citet{Fama2006}
        \item We compute two summary measures of firm strength. The first, OHt, is a mea- sure of bankruptcy risk developed by Ohlson (1980).\citet{Fama2006}
        \item The second composite measure of firm strength, PTt, is from Piotroski (2000).It
        is the sum of nine binary variables, each equal to 1 if a given condition holds and 0 otherwise.\citet{Fama2006}

        \item Financial Performance Signals used to Differentiate high BM Firms \citet{Piotroski2000}
        \item financial variables that reflect changes in these economic conditions should be useful in predicting future firm performance. This logic is used to identify the financial statement signals incorporated in this paper. \citet{Piotroski2000}
        \item I chose nine fundamental signals to measure three areas of the firm's financial condition: profitability, financial leverage/liquidity, and operating efficiency \citet{Piotroski2000}. I define the aggregate signal measure, F-SCORE, as the sum of the nine binary signals. The aggregate signal is designed to measure the overall quality, or strength, of the firm's financial position, and the decision to purchase is ultimately based on the strength of the aggregate signal.
        \item Importance f-score: relative to broader variables capturing changes in the overall health of these companies \cite{Piotroski2000}
        \item This approach represents one simple application of fundamental analysis for identifying strong and weak value firms \citet{Piotroski2000}
        
        \item Measurements (variables) for the financial condition 
    
            \begin{itemize}
            \item industry multiples
            \item Balance sheet ratios
            \item ROE, ROA, accruals 
            \item Fundamental analysis
            \item Working capital adequacy
            \item Asset performance
            \item Capitalization structure
            \end{itemize}
    
        \item Evaluation of the financial strength
    
            \begin{itemize}
            \item Interpretation of the variables
            \item Group investors based on the properties
            \item Pitroskis'f-score as a proxy for financial strength
            \item Use the expected returns of the investors as a proxy for company strength???? If so, the components to measure the expected return can be used to form different groups of investors. Another possibility could be to calculate the expected returns and then form groups of investors? 

            \begin{itemize}
                \item unstable leverage outlook (Deangelo, Harry; Roll, Richard)
                \item components of the f-score (Ball, Ray; Gerakos, Joseph; Linnainmaa, Juhani T.; Nikolaev, Valeri)
                
                
            \end{itemize}
            \item Average return over the last xx years 
            \end{itemize}

    
    \end{itemize}

\subsection{Target}

    \begin{itemize}
        \item Who are the targets - many report that the typical target firm of an activist investor is smaller, more profitable, has a large institutional ownership level, and has more of a “value” orientation (nammely a higher book to market ratio) \citet{CoffeeJr.2014}
        \item the evidence consistently supports only the generalization that targets of activism often tend to have a lower Tobin’s Q and a “value” orientation \citet{CoffeeJr.2014}
        \item For example, Brav, Jiang, Partnoy and Thomas, supra note 8, find no statistically significant relationship between the target’s abnormal returns and their governance and capital structure \citet{CoffeeJr.2014}
        \item These findings are consistent with the idea that the F Score is most effective when applied to stocks for which the market is slow to incorporate relevant financial information. Deep value stocks are typically neglected by analysts and investors and thus likely to exhibit slow impounding of new information \citet{Hyde2014}
        \item the ‘slow impounding of new information’ hypothesis by showing that future institutional investor demand is high for stocks with high F scores. \citet{Hyde2014}
        \item The finding by Mohr (2012) that the F score effectively discriminates between high and low return stocks amongst growth stocks provides additional justification for broadening the analysis to include all stocks. \citet{Hyde2014}
        \item Our key finding is that, consistent with previous evidence from both developed and emerging market studies, stocks with a high F score earn a significant return premium over stocks with a low F score.\citet{Hyde2014}
        \item First, does financial strength predict subsequent institutional demand? \citet{Choi2012}
        \item Analogue to investors? Specifically, the difference between high and low f-score group returns averages 25.73\% (statistically significant at the 1\% level). \citet{Choi2012}
        \item Consistent with Piotroski (2000, 2005) and Fama and French (2006), the results reveal a strong positive relation between f-score 10 and future returns—high f-score stocks average annual market-adjusted returns 8.35\% greater (statistically significant at the 1\% level) than low f-score stocks \citet{Choi2012} 

    \end{itemize}


\section{Underlying Justification for using the given measures as a proxy for financial strength}

    \begin{itemize}
        \item We report that 10-K document file size provides a simple readability proxy that outperforms the Fog Index, does not require document parsing, facilitates replication, and is correlated with alternative readability constructs.\citet{Loughran2014}
        \item Information Availability to the Market
        \item 10-K Filings - MANAGERS OF PUBLICLY traded firms are required to produce public documents
        that provide a comprehensive review of the firm’s business operations and financial condition. An important financial disclosure document created by managers to communicate with investors and analysts is the annual report filed pursuant to the Securities Exchange Act of 1934, Form 10-K.\citet{Loughran2014}
        \item Second, we recommend using the file size of the 10-K as an easily calculated proxy for document readability \citet{Loughran2014}
        \item o ensure that investors would have the necessary information to compute f-scores, Piotroski (2000, 2005) examines annual returns beginning the fifth month following fiscal year end. \citet{Choi2012}
        \item Data availability
        \item Why would they use the data?
        \item Readability: financial information is definitely available - is it readable? (Loughran, MacDonald)
        \item Because information is incorporated in a given manner, we can use several variables as proxies for financial strength observed by the market in the form of abnormal returns (they know what we know)
        \item At the time a company files its 10-K report with the SEC, most likely all key information has already been disclosed to the public.\citet{You2009}
        \item Figure 1 shows the valuation methods2 most widely used by Morgan Stanley Dean Witter’s
        analysts for valuing European companies. Surprisingly, the discounted cash flow (DCF) is in fifth place, behind multiples such as the PER, the EV/EBITDA and the EV/EG \citet{Fernandez2001}
    \end{itemize}

\section{Abnormal Returns - Event Study} 
    \begin{itemize}
        \item Trading strategy of Schedule 13D filers before the filing day \citet{Collin-Dufresne2015}
        \item Figure 2 plots the average buy-and-hold return, in excess of the buy-and-hold return on the value-weightedNYSE/Amex/NASDAQ index from CRSP, from 60 days prior to the filing date to 40 days afterward.\citet{Collin-Dufresne2015}
        \item Short-horizon event studies of stock returns: Many studies have examined what happens to targets firm’s stock price when there is a Schedule 13D filing with the SEC \citet{CoffeeJr.2014}
        \item For ease of exposition, let us define [-x, +y] to be x days before the 13D filing, to y days after the filing. On this basis - literature \citet{CoffeeJr.2014}
        \item event-window: we find that most informed trading before a 13D filing is during the event window (-10, -6).\citet{Brigida2012}
        \item event window - We also found that the target runup before a 13D filing is greatest during the event window (-10,-6). Therefore, future academic research that estimates the share price response surrounding 13D filings should use a window extending to at least 10 days prior to the filing. \citet{Brigida2012}
        \item Outlook: with more information processed one could say... A 13D filing by an acquirer may have a more pronounced impact if the filing specifies that the investor intends to be an activist. \citet{Brigida2012}
        \item Runup = cumulative abnormal return of the target’s stock over the intervals (-10, -1), (-5, -1), and (-2, -1) relative to t = 0 being the filing of the Schedule 13D or 13G. \citet{Brigida2012}
        \item In fact, 92\% of the effect of a Schedule 13D filing on the target’s stock is realized before 3 days prior to the filing. These results show very little new information is revealed to the market when the Schedule 13D filing is made public \citet{Brigida2012}
        \item Therefore, any analysis of the effect of a Schedule 13D filing on the target stock should consider an event window starting no later than ten days before the filing. \citet{Brigida2012}
        \item We define market-adjusted returns as the firm’s buy and hold return less the CRSP value-
        weighted index buy and hold return over the same period. \citet{Choi2012}
        \item computed from the time-series of the differences in the 24 cross-sectional means with Newey-West (1987) standard errors] \citet{Choi2012}
        \item The buy-and-hold Benchmark approach: The first approach uses a benchmark to measure the abnormal buy-and-hold return for every event firm, and tests the null hypothesis that the average abnormal return is zero.\citet{ang2011}
        \item It is well known that event studies are prone to cross-sectional correlation among abnormal returns when the event day is the same for sample firms. \citet{Kolari2010}
        \item We use the buy-and-hold method to measure the abnormal stock returns for two
        reasons. (1) As shown in Barber and Lyon (1997), buy-and-hold is favored over cumulative abnormal return (CAR) on a conceptual ground. (2) BHAR facilitates the cross-sectional analysis of how abnormal return varies with complexity. However, as pointed out by Mitchell and Stafford (2000), BHAR may exaggerate the short-term abnormal return due to compounding. To address this issue, we conduct a calendar time analysis. Specifically, each month we place firms into five portfolios based on their most recent FDR \citet{You2009}
        \item We find that the market reacts favorably to activism, consistent with the view that it creates value. The filing of a Schedule 13D revealing an activist fund’s investment in a target firm results in large positive average abnormal returns, in the range
        of 7\% to 8\%, during the (–20,+20) announcement window \citet{Brav2008}
        \item In the majority of applications, the focus is the effect of an event on the price of a particular class of securities of the firm, most often common equity. \citet{MacKinlay1997}
        \item Useful papers which deal with the practical importance of many of the complications and adjust- ments are the work by Stephen Brown and Jerold Warner published in 1980 and 1985. The 1980 paper considers imple- mentation issues for data sampled at a monthly interval and the 1985 paper deals with issues for daily data \citet{MacKinlay1997}
        \item Also one can easily modify the sta- tistical framework so that the analysis of the abnormal returns is autocorrelation and heteroskedasticity consistent by us- ing a generalized method-of-moments approach.\citet{MacKinlay1997}
        \item The market model represents a potential improvement over the constant mean model \citet{MacKinlay1997}
        \item Restrictions of the CAPM: The use of the Capital Asset Pricing Model is common in event studies of the 1970s \citet{MacKinlay1997} 
        \item mathematics behind it 
        \item Computation etc.
        \item Power functions 
    \end{itemize}

\section{Investor-Target Ratio} 

    \begin{itemize}
        \item These findings are consistent with the idea that the F Score is most effective when applied to stocks for which the market is slow to incorporate relevant financial information. Deep value stocks are typically neglected by analysts and investors and thus likely to exhibit slow impounding of new information \citet{Hyde2014}
        \item What are the characteristics of the targets?
        \item useful signals regarding the likelihood of acquisition
        (Walkling, 1985, and Akhigbe, Martin, and Whyte, 2007) \citet{Brigida2012}
        \item Since a 13D filing can influence the likelihood that a firm will become a takeover target, it is not surprising that the filing elicits a market reaction.\citet{Brigida2012}
        \item What kind of companies are the targets (Coffee Jr., John C.
        Palia, Darius)
        \item relation between ratio and the market 
    \end{itemize}

\section{Data}

    \subsection{Comments}

    \begin{itemize}
        \item Testing the error terms from ordinary least-squares estimations of the below regression equations,using the Breusch-Pagan test, found no evidence for significant heteroscedasticity \citet{Brigida2012}
        \item To reduce any effect of outliers on the estimated coefficients, we also estimate each equation using robust regression employing the Huber weight function. \cite{Brigida2012}
    \end{itemize}

    \subsection{COMPUSTAT - Financial Condition}
    The base accounting variables, from Compustat, are At, total assets (Compustat data
    item 6); Yt, income before extraordinary items (18); ACt, accruals [the change in current assets (4), minus the change in cash and short term investments (1), minus the change in
    current liabilities (5), plus the change in debt in current liabilities (34)]; Dt, total dividends [dividends per share by ex date (26) times common shares outstanding (25)]; and Bt, book equity  [total assets (6), minus liabilities (181), plus balance sheet deferred taxes and
    investment tax credit (35) if available, minus preferred stock liquidating value (10) if available,or redemption value (56) if available, or carrying value (130)]. The accounting variables for year t are measured at the fiscal yearend that falls in calendar year t. \citet{Fama2006}
        \subsubsection{Screening Variables}

        \begin{enumerate}
            \item What consolidation level? - Consolidated
            \item What industry? No financial services (FS) 
            \item What data format?  - Standardized 
            \item Population source? - Domestic 
            \item Currency? - USD 
            \item Company Status? - Active \& Inactive  
        \end{enumerate}

        \subsubsection{Variables}

    \begin{enumerate}
        \item Identifying Information

            \begin{itemize}
                \item Company name 
                \item CIK number 
            \end{itemize}
    
        \item Identifying Information cont.  
        
            \begin{itemize}
                \item GIC variables - GIC sectors etc.
                \item NAICS - in addition to GIC? 
                \item SIC - in addition to GIC?
            \end{itemize}

        \item Company Descriptor

            \begin{itemize}
                \item Acquisition method? - ACQMETH filter by takeover? 
                \item 
            \end{itemize}
        
        \item Balance Sheet Items 

            \begin{itemize}
                \item Current Assets total (ACT)
                \item Total Assets (AT)
                \item Account receivables total (ARTFS)
                \item Cash (CH)
                \item Liabilities total (LT) 
                \item Long term debt total (DLTT)
            \end{itemize}
        
    \end{enumerate}


    \subsection{CRSP - Event Study Abnormal Returns}


\section{Literature}

    \begin{itemize}
        \item For ease of exposition, let us define [-x, +y] to be x days before the 13D filing, to y days after the filing. On this basis \citet{CoffeeJr.2014}
        \item Kothari, S. P., and J. B. Warner. 2007. Econometrics of Event Studies. In B. E. Eckbo (ed.), Handbooks of Corporate Finance: Empirical Corporate Finance, Chapter 1. Amsterdam: Elsevier/North-Holland.
        \item Journal of Financial and Quantitative Analysis (JFQA)
        \item Fama, E. F., and K. R. French. 2006. Profitability, Investment and Average Returns. Journal of Financial Economics 82: 491-518. (page 496)
        \item Mohr, J-H.M., 2012. Utility of Piotroski's F-score for predicting growth stock returns. Working paper, MFIE Capital.
        \item The econometric of Financial Markets - Campbell, Lo and MacKinaly 1997
        \item signs for takeover? 
    \end{itemize}

\end{document}