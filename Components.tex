\documentclass[12pt]{article}
	\usepackage[T1]{fontenc}
	\usepackage[utf8]{inputenc}
	\usepackage[british]{babel}
	\usepackage[a4paper]{geometry}
	\geometry{verbose,tmargin=3cm,bmargin=3.5cm,lmargin=4cm,rmargin=3cm,marginparwidth=70pt}
	\setcounter{secnumdepth}{3}
	\setcounter{tocdepth}{3}
	\usepackage{prettyref}
	\usepackage{textcomp}
	\usepackage{setspace}
	\usepackage{indentfirst}
	\usepackage{fancyhdr}
	\usepackage{url}
	\usepackage[normalem]{ulem}
	\usepackage[table, fixpdftex]{xcolor}
	\usepackage{algpseudocode}
	\usepackage{bigstrut}
	\usepackage{enumitem}

	% package hyperref
	\usepackage{hyperref}
	
	% fancy headers for the thesis
	\fancyhead{}
	\fancyhead[RO]{\slshape \nouppercase \rightmark}
	\fancyfoot[OC]{\begin{flushright}\thepage\end{flushright}}
	\renewcommand{\headrulewidth}{0.4pt}
	\setlength{\headheight}{14pt}

\title{Components of the Thesis}
\author{Leopold Ingenohl}


\begin{document}
\maketitle


\section{Motivation \& Importance} 

    \begin{itemize}
        \item Investor activism 
    \end{itemize}

\section{Descriptive Introduction}

    \begin{itemize}
        \item HNA - Deutsche Bank
    \end{itemize}

\section{Investor \& Target Characteristics}

    \begin{itemize}
        \item Sample selection criteria
    \end{itemize}

\section{Financial Condition}

\subsection{Investor}

    \begin{itemize}
        \item Control variables - implementation?
    
        \item Measurements (variables) for the financial condition 
    
            \begin{itemize}
            \item industry multiples
            \item Balance sheet ratios
            \item ROE, ROA, accruals 
            \item Fundamental analysis
            \item Working capital adequacy
            \item Asset performance
            \item Capitalization structure
            \end{itemize}
    
        \item Evaluation of the financial strength
    
            \begin{itemize}
            \item Interpretation of the variables
            \item Group investors based on the properties
            \item Pitroskis'f-score as a proxy for financial strength
            \item Average return over the last xx years 
            \end{itemize}

    
    \end{itemize}

\subsection{Target}

\section{Underlying Justification for using the given measures as a proxy for financial strength}

    \begin{itemize}
        \item Information Availability to the Market
        \item 10-K Filings? 
        \item Data availability
        \item Why would they use the data?
        \item Readability: financial information is definitely available - is it readable? (Loughran, MacDonald)
        \item Because information is incorporated in a given manner, we can use several variables as proxies for financial strength observed by the market in the form of abnormal returns (they know what we know)
    \end{itemize}

\section{Abnormal Returns - Event Study} 
    \begin{itemize}
        \item event window
        \item mathematics behind it 
        \item Computation etc.
    \end{itemize}

\section{Investor-Target Ratio}  

\section{Data}

    \subsection{COMPUSTAT - Financial Condition}

        \subsubsection{Screening Variables}

        \begin{enumerate}
            \item What consolidation level? - Consolidated
            \item What industry? No financial services (FS) 
            \item What data format?  - Standardized 
            \item Population source? - Domestic 
            \item Currency? - USD 
            \item Company Status? - Active \& Inactive  
        \end{enumerate}

        \subsubsection{Variables}

    \begin{enumerate}
        \item Identifying Information

            \begin{itemize}
                \item Company name 
                \item CIK number 
            \end{itemize}
    
        \item Identifying Information cont.  
        
            \begin{itemize}
                \item GIC variables - GIC sectors etc.
                \item NAICS - in addition to GIC? 
                \item SIC - in addition to GIC?
            \end{itemize}

        \item Company Descriptor

            \begin{itemize}
                \item Acquisition method? - ACQMETH filter by takeover? 
                \item 
            \end{itemize}
        
        \item Balance Sheet Items 

            \begin{itemize}
                \item Current Assets total (ACT)
                \item Total Assets (AT)
                \item Account receivables total (ARTFS)
                \item Cash (CH)
                \item Liabilities total (LT) 
                \item Long term debt total (DLTT)
            \end{itemize}
        
    \end{enumerate}


    \subsection{CRSP - Event Study Abnormal Returns}

    \subsection{Summary Statistics}

\section{Literature}

    \begin{itemize}
        \item Kothari, S. P., and J. B. Warner. 2007. Econometrics of Event Studies. In B. E. Eckbo (ed.), Handbooks of Corporate Finance: Empirical Corporate Finance, Chapter 1. Amsterdam: Elsevier/North-Holland.
        \item Journal of Financial and Quantitative Analysis (JFQA)
        \item Fama, E. F., and K. R. French. 2006. Profitability, Investment and Average Returns. Journal of Financial Economics 82: 491-518. (page 496)
        \item Mohr, J-H.M., 2012. Utility of Piotroski's F-score for predicting growth stock returns. Working paper, MFIE Capital.
        \item signs for takeover? 
    \end{itemize}

\end{document}