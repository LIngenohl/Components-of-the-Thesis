\documentclass[12pt]{article}
	\usepackage[T1]{fontenc}
	\usepackage[utf8]{inputenc}
	\usepackage[british]{babel}
	\usepackage[a4paper]{geometry}
	\geometry{verbose,tmargin=3cm,bmargin=3.5cm,lmargin=4cm,rmargin=3cm,marginparwidth=70pt}
	\setcounter{secnumdepth}{3}
	\setcounter{tocdepth}{3}
	\usepackage{prettyref}
	\usepackage{textcomp}
	\usepackage{setspace}
	\usepackage{indentfirst}
	\usepackage{fancyhdr}
	\usepackage{url}
	\usepackage[normalem]{ulem}
	\usepackage[table, fixpdftex]{xcolor}
	\usepackage{algpseudocode}
	\usepackage{bigstrut}
	\usepackage{enumitem}

	% package hyperref
	\usepackage{hyperref}
	
	% fancy headers for the thesis
	\fancyhead{}
	\fancyhead[RO]{\slshape \nouppercase \rightmark}
	\fancyfoot[OC]{\begin{flushright}\thepage\end{flushright}}
	\renewcommand{\headrulewidth}{0.4pt}
	\setlength{\headheight}{14pt}

\title{TO-DO}
\author{Leopold Ingenohl}


\begin{document}
\maketitle

\section{Literature}
\begin{itemize}
   \item Kothari, S. P., and J. B. Warner. 2007. Econometrics of Event Studies. In B. E. Eckbo (ed.), Handbooks of Corporate Finance: Empirical Corporate Finance, Chapter 1. Amsterdam: Elsevier/North-Holland.
   \item Journal of Financial and Quantitative Analysis (JFQA)
   \item Fama, E. F., and K. R. French. 2006. Profitability, Investment and Average Returns. Journal of Financial Economics 82: 491-518. (page 496)
   \item Mohr, J-H.M., 2012. Utility of Piotroski F-score for predicting growth stock returns. Working paper, MFIE Capital.
   \item signs for takeover? 
   

\end{itemize}

\section{Data}
\subsection{Sample Criteria}
\begin{itemize}
    \item Filtered by 10-K filings
    \item financial information is definitely available - is it readable? (Loughran, MacDonald)
    \item 
\end{itemize}
\section{Summary statistics}

\section{Compustat}
\begin{enumerate}

    \item Screening Variables
   
    \begin{enumerate}
        \item What consolidation level? - Consolidated
        \item What industry? No financial services (FS) 
        \item What data format?  - Standardized 
        \item Population source? - Domestic 
        \item Currency? - USD 
        \item Company Status? - Active \& Inactive  
    \end{enumerate}
   
    \item Variables 
    
    \begin{enumerate}
        \item Identifying Information
        \begin{itemize}
            \item Company name 
            \item CIK number 
        \end{itemize}
    
        \item Identifying Information cont.  
        \begin{itemize}
            \item GIC variables - GIC sectors etc.
            \item NAICS - in addition to GIC? 
            \item SIC - in addition to GIC?
        \end{itemize}

        \item Company Descriptor
        \begin{itemize}
            \item Acquisition method? - ACQMETH filter by takeover? 
            \item 
        \end{itemize}
        
        \item Balance Sheet Items 
        \begin{itemize}
            \item Current Assets total (ACT)
            \item Total Assets (AT)
            \item Account receivables total (ARTFS)
            \item Cash (CH)
            \item Liabilities total (LT) 
            \item Long term debt total (DLTT)
        \end{itemize}
        
    \end{enumerate}
    
    \item asdasd

\end{enumerate}


\section{Financial Strength}

    \begin{itemize}
    \item Control variables - implementation?

    \item Measurements (variables) for the financial condition 

        \begin{itemize}
        \item industry multiples
        \item Balance sheet ratios
        \item ROE, ROA, accruals 
        \item Fundamental analysis
        \item Working capital adequacy
        \item Asset performance
        \item Capitalization structure
        \end{itemize}

    \item Evaluation of the financial strength

        \begin{itemize}
        \item Interpretation of the variables
        \item Group investors based on the properties
        \item Pitroskis'f-score as a proxy for financial strength
        \item Average return over the last xx years 
        \end{itemize}

    \item 
    \end{itemize}

\section{Information used by investors}

\begin{itemize}
    \item 10-K Filings? 
    \item Data availability
    \item Why would they use the data?
    \item Because information is incorporated in a given manner, we can use several variables as proxies for financial strength (they know what we know)
\end{itemize}
\end{document}